% Use the standard \LaTeXe\ article style in 12pt Computer Modern font 
% on A4 paper by
\documentclass[12pt,a4paper]{article}
% Do \emph{not} change the width nor the height of the text from the 
% defaults set by this document class.
% 
% The alternative which is closer to what we actually use is
% \documentclass[11pt,a5paper]{article}
% \usepackage[a5paper]{geometry}
% Because it is a great size for on screen reading
% and prints nicely on a4paper either 2up or booklet.

% The preamble is to contain your own \LaTeX\ commands and to say 
% what packages to use.  Three useful packages are the following:
\usepackage{graphicx} % avoid epsfig or earlier such packages
\usepackage{url}      % for URLs and DOIs
\newcommand{\doi}[1]{\url{http://dx.doi.org/#1}}
\usepackage{hyperref}
\usepackage{amsmath}  % many want amsmath extensions
\usepackage{amsfonts}
\usepackage[usenames]{color}
\usepackage[ruled,vlined]{algorithm2e}

\usepackage{amsthm}
\usepackage{amssymb}
\usepackage{pgfplots}
%\pgfplotsset{compat=1.3}

% Avoid loading unused packages (as done by some \LaTeX\ editors).

%\input{psfig.sty}
%
\newtheorem*{lemma}{Lemma}
\newtheorem{Lemma}{Lemma}
\newtheorem*{theorem}{Theorem}
\newtheorem{Theorem}[Lemma]{Theorem}
\newtheorem*{conjecture}{Conjecture}
\newtheorem{Conjecture}[Lemma]{Conjecture}
\newtheorem*{corollary}{Corollary}
\newtheorem{Corollary}[Lemma]{Corollary}
\newtheorem*{remark}{Remark}
\newtheorem{Remark}[Lemma]{Remark}
\newtheorem*{definition}{Definition}
\newtheorem{Definition}{Definition}

\newcommand{\Integer}{\mathbb Z}
\newcommand{\Natural}{\mathbb N}
\newcommand{\Real}{\mathbb R}
\newcommand{\Complex}{\mathbb C}

\newcommand{\Domain}{\mathcal D}
\newcommand{\RKHS}{\mathcal H}

\newcommand{\Hilbert}{\mathbb H}
\newcommand{\Poly}{\mathbb P}
\newcommand{\Sphere}{\mathbb S}
\newcommand{\Torus}{\mathbb T}

\newcommand{\Kernel}{\mathcal K}
\newcommand{\kernel}{K}

\newcommand{\One}{\mathbf 1}

\newcommand{\WTP}{\operatorname{WTP}}
\newcommand{\norm}[1]{\left\Vert #1 \right\Vert}
\newcommand{\OrderH}{\operatorname{O}}
\newcommand{\OrderL}{\operatorname{\Omega}}
\newcommand{\Pt}[1]{\mathbf{#1}}
\newcommand{\To}{\rightarrow}

\newcommand{\Prog}[1]{\texttt{#1}}

\newcommand{\argmin}{\operatorname{argmin}}
\newcommand{\argmax}{\operatorname{argmax}}
\newcommand{\linspan}{\operatorname{span}}

\newcommand{\mb}[1]{\mathbb{#1}}
\newcommand{\mf}[1]{\mathbf{#1}}

% Create title and authors using \verb|\maketitle|.  Separate authors by 
% \verb|\and| and put addresses in \verb|\thanks| command with
% \verb|\url| command \verb|\protect|ed.

\title{Parallel enumeration of word overlap correlation classes using MPI}

\author{
Paul~C.~Leopardi
\thanks{Centre for Mathematics and its Applications, Australian National University.
\protect\url{mailto:paul.leopardi@anu.edu.au}}
}

\date{}

\begin{document}

\maketitle

\begin{abstract}
I now have an MPI version of the word-overlap-correlation program running on Orac. 
At the moment, each job runs on 8 cores, as 6 ``worker'' processes and 2 ``scribe'' processes. 
The workers enumerate Restricted Growth Strings and transform these into 
correlation classes for pairs of words. 
The scribe processes record and count the correlation classes.
This note describes some aspects of the organization of the code and how it uses MPI.
\end{abstract}
% By default we include a table of contents in each paper.
%\tableofcontents

\section*{Splitting the sequence of beta-restricted growth strings into sub\-sequences}

The program \texttt{word-over-corr-mpi-worker-scribe} enumerates a sequence of 
beta-RGS (restricted growth strings for set partitions into beta parts) 
in parallel, by splitting it into subsequences. Each subsequence is defined
by two beta-RGS, which are instantiated by \texttt{ulong} arrays, called \verb!rgs_begin! 
and \verb!rgs_end!, in the functions \verb!check_setpart_p_rgs_range()! and \verb!main()!. 
Each subsequence begins with \verb!rgs_begin!, which is used to instantiate an 
object of class \verb!setpart_p_rgs_lex! via the following call:
\begin{verbatim}
 setpart_p_rgs_lex p = setpart_p_rgs_lex(2*T,beta,rgs_begin);
\end{verbatim}

The function \verb!check_setpart_p_rgs_range()! contains a loop with the structure:

\begin{verbatim}
  for (bool more=true; more; more=p.next())
  { ...
    const ulong* rgs_data = p.data();
    if (arrays_are_equal(2*T, rgs_data, rgs_end))
      break;
    ...
  }
\end{verbatim}

Thus the subsequence begins with \verb!rgs_begin!, and continues via \verb!p.next()!, 
up to but not including \verb!rgs_end!. A special sentinel value is used for 
\verb!rgs_end! in the case of the final subsequence.

The program currently uses two methods to create the beta-RGS \texttt{ulong} arrays.

The first method is via the function \verb!new_setpart_p_rgs_array()! as follows:

\begin{verbatim}
  ulong* new_setpart_p_rgs_array(const ulong beta, 
                                 const ulong gamma)
  {
    ulong* s = new ulong[2*T]; ...
    for (ulong k = 0; k != 2*T; k++) 
      s[k] = 0UL;
    if (gamma > beta)
      s[0] = ~0UL;
    else
    {
      for (ulong k=0; k != gamma; k++)
        s[k] = k;
      for (ulong k=0; gamma + k < beta; k++)
        s[2*T-beta+gamma+k] = gamma+k;
      ...
    }
    return s;
  }
\end{verbatim}

This routine can create \verb!beta+1! different beta-RGS arrays, depending on the
value of \verb!gamma!. 

The second method is to call 

\begin{verbatim}
  setpart_p_rgs_lex q = setpart_p_rgs_lex(delta,beta);
\end{verbatim}

with \verb!delta! $\geqslant$ \verb!beta!, continue via \verb!q.next()!, and extend the array by
appending zeros. In the function \verb!main()!, this looks like:

\begin{verbatim}
  ulong delta = (beta+WOC_DELTA_OFFSET > 2*T) 
              ? beta : beta+WOC_DELTA_OFFSET;
  setpart_p_rgs_lex r = setpart_p_rgs_lex(2*T,beta);
  const ulong* r_data = r.data();
  ulong* rgs_begin = new ulong[2*T]; ...
  for (int i = 0; i != 2*T; i++) 
    rgs_begin[i] = r_data[i];
  ulong* rgs_end = new ulong[2*T]; ...
  for (int i = delta; i < 2*T; i++)
    rgs_end[i] = 0;
  setpart_p_rgs_lex q = setpart_p_rgs_lex(delta,beta); ...
  for (bool more = true; more; more = q.next(), ...)
  {
    const ulong* q_data = q.data();
    for (int i = 0; i != delta; i++)
      rgs_end[i] = q_data[i];
    // The subsequence [rgs_begin,rgs_end) 
    // has now been defined.
    for (int i = 0; i != 2*T; i++)
      rgs_begin[i] = rgs_end[i];
  }
  rgs_end[0] = ~0UL;
  // The final subsequence [rgs_begin,rgs_end)
  //  has now been defined.
\end{verbatim}

I call this second method a double-loop method. The outer loop
enumerates beta-RGS of length \verb!delta!, 
where \verb!beta! $\leqslant$ \verb!delta! $\leqslant$ \verb!2*T!.
The inner loop enumerates beta-RGS of length \verb!2*T!.

%%%%%%%%%%%%%%%%%%%%%%%%%%%%%%%%%%%%%%%%%%%%%%%%%%%%%%%%%%%%%%
\paragraph{Acknowledgements.}
%\begin{acknowledgement}

Thanks to Joerg Arndt for the FXT library and many useful discussions.

%\end{acknowledgement}
%%%%%%%%%%%%%%%%%%%%%%%%%%%%%%%%%%%%%%%%%%%%%%%%%%%%%%%%%%%%%%%%%%%%%%

% Choose one of these bibliography styles, comment out the other:

%\addtocontents{toc}{\vspace{0.5cm}}

\bibliographystyle{abbrv}

%\bibliography{sph}

\end{document}
% ----------------------------------------------------------------
